% !TEX root = 0.gesamt.tex

% \newcolumntype{L}[1]{>{\raggedright\arraybackslash}p{#1}} % linksbündig mit Breitenangabe

\begin{landscape}
%\thispagestyle{empty}
\section{Verlaufsplan}

\singlespacing
\scriptsize
\begin{longtable}{p{2cm}|p{5.5cm}|p{5.5cm}|p{2cm}|p{2cm}|p{4cm}}

\textbf{Phase} &
\textbf{Aktivität Lehrkraft} &
\textbf{Aktivität SuS} &
\textbf{Sozialform} &
\textbf{Medien, Material} &
\textbf{didaktisch-methodischer Kommentar}
\\
\hline
\textbf{Einstieg}
&
- begrüßt SuS und Gäste \newline
- bittet SuS um Zusammenfassung der letzten Stunde und notiert die genannten Kriterien an Tafel \newline
- nennt das heutige Thema, verweist auf die notierte Tagesordnung \newline
- teilt die Paare ein \newline
- erläutert das Aufwärmspiel \newline
&
- begrüßen L und Gäste \newline
- nennen wesentliche Inhalte der letzten Stunde \newline
- wiederholen Kriterien des \emph{Innenspannstoßes}\newline
&
Plenum / UG
&
Tafel, Kreide/Stift, Smartphone, TeamShake
&
jede Stunde startet im Klassenzimmer, strukturierter Beginn

\\
\hline
\textbf{Aufwärmen}
&
bestimmt neue Fänger-Pärchen (ggf. auch zwei)
&
spielen
&
Spiel, PA
&
2-3 Bälle
&
Erwärmung dient auch der Wiederholung (statisches Passen) und der Vorbereitung auf den Hauptteil (PA)
\\
\hline

\textbf{Erarbeitung I}
&
- \emph{Was fehlt noch zum \enquote{richtigen} Passen?} \newline
- erläutert die Aufgabe, skizziert Aufbau und mögliche Pass- und Laufwege, demonstriert ggf. einmal mit eine*r Schüler*in die Aufgabe \newline
- beaufsichtigt, gibt Hinweise und Start-/Stopp Signale
&
- nennen \emph{Bewegung} als fehlendes Element \newline
- finden sich in 4er-Gruppen zusammen \newline
- jedes Paar durchläuft zweimal für 2 Minuten den \enquote{Wald}, das jeweils andere Paar beobachtet und gibt Hinweise auf Basis der Kriterien und zählt die durchpassten Tore (im 2. Durchgang)
&
Plenum / UG, PA, GA
&
Tafel, Floormarker, 7-8 Fußbälle
&

\\
\hline
\textbf{Zwischen"-be"-sprechung}
&
- lobt die SuS und fragt nach Schwierigkeiten \newline
- \emph{Was könnte man machen, um schneller und sicherer durch den Wald zu kommen?} \newline
-  stellt den Beobachtungsbogen vor \newline
&
nennen: in Bewegung bleiben (so wenig wie möglich stehen), präziser passen, mehr passen
&
Plenum
&
Beobach"-tungs"-bogen
&
\\
\hline
\textbf{Erarbeitung II} \newline
&
unterstützt die Coaches und die Beratung der Spieler*innen
&
- laufen mindestens 2x den Parcours  \newline
- Coaches beobachten, zählen, notieren, beraten
&
PA, GA
&
Floormarker, Beobachtungsbögen, Klemmbretter, Kugelschreiber, 7-8 Fußbälle
&
- die Coaches wechseln nach dem ersten Durchgang die Aufgaben (Zählen, Notieren/Beobachten) \newline
- Differenzierung: beidfüßig, langsamer

\\
\hline
\textbf{Abschluss}
&
- Ergebnisvergleich: wer hat die meisten Tore geschafft? \newline
- Was von den genannten Kriterien ging schon gut, was wollen wir noch üben? \newline
- Wie war das Arbeiten mit den Coaches? \newline
- bedankt sich und verabschiedet die SuS
&
- nennen Ergebnisse \newline
- benennen noch einmal die Kriterien und bewerten das bisher Erlernte sowie die Methode der \emph{Personal Coaches} \newline
- geben Stimmungsbild per Daumenprobe
&
Plenum (Klassenzimmer)
&
Tafel, Stift oder Kreide
&

\end{longtable}
\end{landscape}

\normalsize
