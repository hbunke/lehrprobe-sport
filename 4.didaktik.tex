% !TEX root = 0.gesamt.tex
\section{Didaktische Analyse}

\textbf{Didaktische Leitidee:} Die SuS kombinieren das in der letzten Stunde
erlernte bzw. verfestigte statische Passen mit Laufbewegungen auf engem Raum.
Dabei wird die Komplexität durch das Abstimmen mit der Partnerin / dem Partner
sowie dem Ausweichen weiterer übender Pärchen erhöht und die Bewältigung durch
kriterienorientierte kognitive Reflexion mit \enquote{Personal Coaches} und in
der Lerngruppe unterstützt. Übungsanreiz bietet ein Wettbewerb, in dem die SuS
in begrenzter Zeit möglichst viele Pässe durch vorgegebene Ziele spielen
sollen.

Nimmt man die Grundfragen der didaktischen Analyse nach \citeA{klafki1962} zum
Maßstab, ist zunächst die hohe \textbf{Gegenwartsbedeutung} des Themas Fußball
hervorzuheben. Fußball als gesellschaftlich, kulturell und sozial tief
verankerte Sportart bietet hohes soziales Integrations- und
Interaktionspotenzial. Sie auf auch nur einem Basis-Niveau zu verstehen und
beherrschen, befähigt zu gesellschaftlicher und sozialer Teilhabe. Dies gilt
für alle SuS, ganz besonders jedoch -- bezogen auf die Lerngruppe der
vorliegenden Stunde -- für die Mädchen. Während eine Fußballsozialisation für
Jungen weitgehend selbstverständlich, zumindest aber leicht zugänglich ist,
haben Mädchen es diesbezüglich aus verschiedenen Gründen immer noch erheblich
schwerer. Speziell für sie besteht die \textbf{Zukunftsbedeutung} des Themas
deshalb auch darin, die Chancen und Möglichkeiten, die der Fußball bietet, für
sich genauso nutzen zu können, wie Jungen das können. Man kann in Anlehnung an
einen Begriff aus der Sozialpsychologie von \enquote{Empowerment} sprechen, was
u.a. auch das Erwerben von Kompetenzen zur gesellschaftlichen und kulturellen
Teilhabe umfasst. \enquote{Bezogen auf Mädchenfußball heißt das, dass Mädchen
dazu ermuntert werden sollen in bisher männerdominierte Welten einzudringen und
sich dort zu behaupten} \cite[S. 13]{kugelmann2009}. In naher Zukunft ist in
diesem Kontext für die UE auch die vom 7.6.-7.7.2019 in Frankreich
stattfindende Fußball-Weltmeisterschaft der Frauen bedeutsam, die den
Schülerinnen im Anschluss an die Unterrichtseinheit unmittelbar
Identifikations- und Anwendungsmöglichkeiten bietet.

Das Thema \emph{Passen} als Schwerpunkt der vorliegenden wie auch der
vorhergehenden Stunde ist prinzipiell nicht nur für Fußball, sondern für nahezu
alle Ballsportarten von grundlegender Bedeutung. So kann denn auch die in
dieser Stunde schwerpunktmäßig durchgeführte Übung hinsichtlich der Laufwege,
der Raumorientierung, dem Bewegen mit Ball und beim Passen nahezu 1:1 auf
andere Sportspiele übertragen werden (\textbf{exemplarische Bedeutung}).
\textbf{Zugänglich} wird den SuS das Passen in Bewegung zunächst durch die
Bewusstmachung, dass in realen Spielsituationen so gut wie nie nur im Stand
gepasst wird und -- aufgrund von Gegnereinwirkungen -- auch nicht werden kann.
Darauf aufbauend wird auch das Thema Geschwindigkeit (Gegner überlaufen,
überpassen) angesprochen. Motivational unterstützt wird dies durch den
Wettbewerb (\emph{wer schafft die meisten Tore in 2 Minuten zu durchspielen?}).
Geschwindigkeit stellt auch die erste Ebene der \textbf{inhaltlichen
Strukturierung} dar. Weitere sind Pass-Präzision sowie die Koordination (Raum)
und Absprache mit der/dem Mitspieler*in.
