% !TEX root = 0.gesamt.tex

% background color for special row in unterrichtssequenz
\definecolor{Gray}{gray}{0.9}

\section{Einordnung des Themas in curriculare Vorgaben und die Unterrichtssequenz}

Die Unterrichtseinheit \emph{Fußball} ordnet sich in das im Bildungsplan für
Gymnasien aufgeführte Bewegungsfeld \emph{Spielen} und den dort genannten
Sportbereich \emph{große Sportspiele} \cite[S.
10]{dersenatorfurbildungundwissenschaft2006a}. Im schulinternen Curriculum wird
als Unterrichtsvorhaben in den Klassen 5-6 für das Bewegungsfeld \emph{Spielen}
\enquote{Erste Einführung der Großen Spiele} gelistet \cite[S.
2]{kippenberg-gymnasiumbremen2011}. Als Kompetenzen werden in der vorliegenden
Stunde besonders gefördert \enquote{mit- und gegeneinander spielen},
\enquote{sich im Raum orientieren}, \enquote{partnerschaftlich agieren} sowie
\enquote{bisher erworbene allgemeine Spielfähigkeiten in das entsprechende
Spielgeschehen einbringen} \cite[S.
11]{dersenatorfurbildungundwissenschaft2006a}.

\subsection{Unterrichtssequenz}
\begin{scriptsize}
\begin{singlespacing}

\begin{longtable}{|p{3.5cm}|p{4.5cm}|p{7cm}|}
\hline
\textbf{Thema} & \textbf{Inhalt} & \textbf{didaktische Ziele / Kompetenzen}

\\
\hline
Der Ball ist dein Freund!
&
Ballgewöhnung und Dribbeln
&
Vorwissen Fußball aktivieren; Ballgewöhnung; Ball am Fuß führen; Wahrnehmung spezieller Fußballeigenschaften

\\
\hline
Den Ball zuspie"-len
&
Einführung Passen, Stoppen (statisch)
&
Innenspannstoß; korrekte Fußhaltung; präziser Pass zu Partner*in

\\
\hline
\rowcolor{Gray}
\textbf{Den Ball in Bewegung zuspielen}
&
\textbf{Passen in Bewegung}
&
\textbf{Anwendung Innenspannstoß in spielnaher Situation; Beobachtung und Beratung;}

\\
\hline

Tore schießen
&
Torschuss und Schusstechniken (Innenspann, Vollspann)
&
präzise schießen; kraftvoll und entschlossen schießen; schießen aus unterschiedlichen Spielsituationen

\\
\hline

Den Ball erobern
&
Zweikampf und Körpereinsatz
&
den Körper (fair) einsetzen; Körperkontakt; Körperhaltung; Ängste überwinden

\\
\hline

Zusammen spielen
&
Raumorientierung und Individualtaktik
&
Wahrnehmung von Raum und Mitspieler*innen; Positionierung im Raum; Freilaufen
\\
\hline

\end{longtable}
\end{singlespacing}
\end{scriptsize}



\vspace{0.5cm}
