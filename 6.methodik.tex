% !TEX root = 0.gesamt.tex
\section{Methodische Überlegungen}

Methodisch im Zentrum der Stunde steht eine einzige Übung, das Passen mit
einem/einer Partner*in durch einen mit Floormarkern\footnote{Es werden hier
keine sonst üblichen Pylonen oder einfache Hütchen verwendet, da diese auf
Hallenboden speziell bei Fußballübungen leicht verrutschen oder umfallen.
Floormarker sind auch auf Hallenboden rutschfest und gleichzeitig gut
sichtbar.} markierten \enquote{Wald} oder Parcours. Die Konzentration auf eine
Übung ermöglicht zum einen motorische und kognitive Vertiefung, zum anderen
kann durch die Beibehaltung eines sehr einfachen Aufbaus Zeit für die
Erarbeitung gewonnen werden. Zweiter methodischer Kern-Aspekt ist, in Anlehnung
an das Konzept des \enquote{Personal Coaching} \cite{flemming2009}, die
einfache  kriteriengestützte Beobachtung und Beratung der SuS untereinander.
Dies ermöglicht zum einen eine angesichts der geringen Größe der Halle
notwendige organisatorische Teilung der Lerngruppe, ohne dass eine Gruppe
passiv nur zuschauen muss \cite<vgl.>[S. 13]{wagener2018}. Zum anderen fördert
es zusätzlich zur technischen Fachkompetenz \enquote{kooperatives Verhalten}
und \enquote{methodisches Denken} \cite[S. 20]{flemming2009}.

Nach der Begrüßung der SuS im ritualisiert eingeführten \enquote{Klassenzimmer}
werden zunächst zur \textbf{Aktivierung des Vorwissens} die in der letzten
Stunde erarbeiteten wesentlichen Merkmale des Innenspannstoßes wiederholt und
an der Tafel notiert. Sie sind auch Bestandteil der Beobachtungsaufgaben im
Hauptteil. Nachfolgend teilt die Lehrkraft mittels
\emph{TeamShake}\footnote{\emph{TeamShake} ist eine App, mit der sich leicht
Guppen für Sportspiele einteilen lassen. Dabei geschieht die Einteilung nicht
nur zufallsbasiert, sondern kann nach Stärken, Geschlecht etc. konfiguriert
werden.} die Partner für das folgende Aufwärmspiel zusammen. Die Paarungen
bleiben für den Rest der Stunde bestehen und werden im Hauptteil zu 4-er
Gruppen zusammengesetzt.

Das folgende Spiel dient der allgemeinen und speziellen \textbf{Erwärmung},
aber auch der Wiederholung des in der letzten Stunde erlernten statischen
Passens und der Partnergewöhnung.\footnote{Ein Paar agiert als Fängergruppe,
die anderen laufen jeweils Hand in Hand durch die Halle. Ein Paar, das getickt
wurde, macht auf dem Boden zwei Brücken hintereinander. Durch dieses
\enquote{Tor} muss der Ball von einem anderen Paar einmal durchgepasst werden,
um die Gefangenen zu befreien.}

Der Einstieg in die \textbf{Erarbeitungsphase} beginnt mit einer Frage der
Lehrkraft, mit der das bisher Gelernte mit den Vorstellungen der SuS vom
\enquote{richtigen} Fußballspielen abgeglichen werden. Die SuS erkennen
\enquote{Bewegung} als wesentlichen Unterschied. Ausgehend davon erfolgt die
Erläuterung von Ablauf und Zweck der folgenden Übung. Zur visuellen
Unterstützung werden Parcours, Lauf- und Passwege auf der Tafel skizziert. Der
Zusammenschluss der Paare zu Vierergruppen erfolgt durch die SuS, da für die
Bewertungsaufgaben ein gewisses Vertrauensverhältnis nötig ist
\cite<vgl.>{flemming2009}. Im ersten Durchgang der Übungsphase durchläuft jedes
Paar einmal für zwei Minuten den Parcours und wird dabei beobachtet von dem
zweiten Paar der 4er-Gruppe, den \emph{Coaches}. Hier liegt der Schwerpunkt
zunächst auf der Gewöhnung, die SuS sollen vor allem auf die Umsetzung der
erlernten Passtechnik sowie die enge Ballführung achten. Den Gruppen wird nach
jedem Durchgang kurz Gelegenheit zur Besprechung gegeben. In einem folgenden
zweiten Durchgang sollen dann die durchpassten Tore gezählt werden, um zum
einen einen Ansporn zu bieten und um zum anderen später Vergleichsmöglichkeiten
zu haben. Diese Aufgabe übernehmen die Coaches. Auch hier wird wieder kurz
Gelegenheit zum Besprechen gegeben. In einer gemeinsamen
\textbf{Zwischenbesprechung} wird zum einen reflektiert (\emph{was ist euch
aufgefallen?}), zum anderen werden Kriterien gesammelt, die die Bewältigung des
Parcours sicherer und schneller machen können (in Bewegung bleiben, präziser
passen, mehr passen). Diese Kriterien finden sich auch in den Bewertungsbögen
für die Coaches wieder, von denen jedes Paar (inkl. Klemmbrett und
Kugelschreiber) nun einen erhält. In zweiten Teil der Erarbeitung wird nun das
Durchlaufen des Parcours von den Coaches mit Hilfe der genannten Kriterien
bewertet, auf den Bögen notiert und anschließend mit der \enquote{Mannschaft}
besprochen. Letztere durchläuft dann mit den Tipps der Coaches noch einmal den
\enquote{Wald}. Nach einer weiteren kurzen Besprechungsphase zwischen Coaches
und Mannschaften erfolgt der Wechsel. Zur \textbf{Differenzierung} kann in
dieser Phase besonders leistungsstarken SuS der Auftrag gegeben werden,
beidfüßig zu passen (abwechselnd links/rechts) oder auch dein Außenspannpass zu
probieren. SuS mit Problemen beim Passen kann ein Punktebonus gegeben werden
und sie werden angehalten, statt auf Tempo vor allem auf Präzision und
Absprache zu achten. In der abschließenden \textbf{Auswertung und Sicherung}
nach dem Abbau des Parcours fragt die Lehrkraft zunächst nach den maximal
durchspielten Toren (\textbf{Wettbewerb}) und fragt dann die SuS, was von den
genannten Kriterien gut klappt und wo noch geübt werden sollte. Die Ergebnisse
werden an der Tafel gesichert. Daneben werden die Bewertungsbögen (die anonym
bleiben) eingesammelt und später von mir ausgewertet.

Als \textbf{didaktische Reserve} kann mit demselben Aufbau auf zwei Feldern ein
Blitzturnier mit den bereits eingeteilten 4er-Gruppen gespielt werden. Ziel ist
das Durchpassen eines der Floormarker-Tore, bei einem Treffer werden die
Mannschaften gewechselt.
