% !TEX root = 0.gesamt.tex
\section{Sachanalyse}

Fußball ist die weltweit beliebteste und am häufigsten betriebene Ballsportart
überhaupt. Die Gründe dafür sind vielfältig und reichen von der Einfachheit der
Regeln und des Spielgedankens über die Ritualisierung und gesellschaftliche
Verankerung bis hin zur sozialen Funktion \cite<vgl. dazu knapp
zusammenfassend>[S. 255f.]{stiehler1988}. Fußball gehört daher nach wie vor
auch fest in den Kanon schulischer Sportarten, schon weil
\enquote{Lebensweltbezug und Aktualität doch direkt gegeben} sind \enquote{und
nicht konstruiert werden} müssen \cite[S. 9]{lutgerodt2018}.

Das \emph{Passen} gehört neben dem Dribbeln und Schießen zu den drei
wesentlichen Grundtechniken des Fußballspiels. Betrachtet man die bis in den
Kinder- und Jugendbereich wirkende internationale Entwicklung des Fußballs in
den letzten zehn Jahren, hat das Passen als Grundelement des Ballbesitzfußballs
einen enormen Bedeutungszuwachs erfahren \cite<vgl.>[Kap. 2]{hyballa2014}. Und
auch wenn natürlich solche Anforderungen nicht in den Bereich von Kindern und
Jugendlichen übertragbar sind \cite[S. 226.f]{schomann2016}, kann das technisch
saubere Passen des Balls inzwischen als die wichtigste zu erlernende Technik
betrachtet werden, um im Fußball eine spezielle Spielfähigkeit zu
erlangen.\footnote{Dies spiegelt sich auch in den DFB-Schulungsmaterialien
wider, siehe z.B. \citeA[S. 29ff.]{kuhlmann2016}.}

Der \emph{Innenspannstoß} ist die am häufigsten angewandte Passtechnik und in
seiner statischen Form relativ einfach zu erlernen: Schussbein aufgedreht, Ball
mit dem Zentrum der Innenseite treffen, Standbein steht neben Ball;
\cite{schomann2016}. Die Anforderungen werden jedoch komplexer, wenn in
Bewegung gepasst wird. Hier kommt die Abstimmung mit dem Passempfänger hinzu,
Laufwege müssen antizipiert werden und die Passausführung wie auch die
Ballannahme müssen mit der eigenen Laufbewegung koordiniert werden.
